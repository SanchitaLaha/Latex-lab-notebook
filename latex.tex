\documentclass[12pt,a4paper]{article}
\usepackage{graphicx}
\usepackage{amsmath}
\usepackage{xcolor}
\usepackage{geometry}
\usepackage{fancyhdr}
\usepackage{tikz}
\usepackage{enumitem}

\geometry{a4paper, margin=1in}

% Set up the header and footer
\pagestyle{fancy}
\fancyhf{}
\fancyhead[L]{\leftmark}
\fancyhead[R]{\thepage}

\begin{document}
	
	% Title Page
	\begin{titlepage}
		\centering
		\vspace*{2cm}
		
		% Decorative Line
		\noindent\rule{\textwidth}{1pt}\\[0.5cm]
		
		% Project Title
		{\Huge\textbf{\color{blue}LAB NOTEBOOK USING LATEX}}\\
		\vspace{0.5cm}
		\noindent\rule{\textwidth}{1pt}\\[2cm]
		
		% Group Details
		\Large\textbf{Submitted by:}\\
		\vspace{0.5cm}
		
		\textbf{\color{red}Group Leader:}\\
		\vspace{0.3cm}
		\Large MD Erfanuddin\\
		\large Roll Number: 30001223001\\
		Department: BCA\\
		
		\vspace{1.5cm}
		
		\textbf{\color{green}Group Members:}\\
		\vspace{0.5cm}
		
		\large
		\begin{tabbing}
			\hspace{4cm} \= \hspace{8cm} \= \kill
			\textbf{Name} \> \textbf{Roll Number} \> \textbf{Department} \\
			\textcolor{purple}{Sanchita Laha} \> 30084323022 \> BSc in Data Science \\
			\textcolor{purple}{Arnov Pal} \> 30001223067 \> BCA \\
			\textcolor{purple}{Abir Dey} \> 30001223010 \> BCA \\
			\textcolor{purple}{Ibnoon Qais} \> 30059223003 \> BSc in Forensic Sciences \\
		\end{tabbing}
		
		\vfill
		
		% Department and University Information
		\Large\textbf{\color{orange}Git Assignment}\\
		\Large\textbf{\color{blue}Maulana Abul Kalam Azad University of Technology}\\
		\vspace{0.5cm}
		\Large\textbf{Date: \today}
		
		% Decorative Line at the Bottom
		\vspace{2cm}
		\noindent\rule{\textwidth}{1pt}
		
	\end{titlepage}
	
	% Index Page
	\newpage
	\tableofcontents
	
	% Md Erfanuddin's Page
	\newpage
	\section{C Programming Calculator /by : Md.Erfanuddin}
	\subsection{Introduction}
	This document outlines how to set up a local repository, create a C calculator program, commit the code, and push it to GitHub.
	\subsection{Steps}
	\begin{enumerate}
		\item Create a local repository using git init.
		\item Build the C program (Calculator).
		\item Commit the code with git commit.
		\item Create a public repository on GitHub.
		\item Push the local repository to GitHub.
	\end{enumerate}
	\subsection{Code Snippet}
	\begin{verbatim}
		#include <stdio.h>
		
		int main() {
			char operator;
			double num1, num2;
			
			printf("Enter an operator (+, -, *, /): ");
			scanf("%c", &operator);
			printf("Enter two operands: ");
			scanf("%lf %lf", &num1, &num2);
			
			switch (operator) {
				case '+': printf("%.2lf + %.2lf = %.2lf\n", num1, num2, num1 + num2); break;
				case '-': printf("%.2lf - %.2lf = %.2lf\n", num1, num2, num1 - num2); break;
				case '*': printf("%.2lf * %.2lf = %.2lf\n", num1, num2, num1 * num2); break;
				case '/': 
				if (num2 != 0.0) printf("%.2lf / %.2lf = %.2lf\n", num1, num2, num1 / num2);
				else printf("Division by zero error.\n");
				break;
				default: printf("Invalid operator.\n"); break;
			}
			return 0;
		}
		
	\end{verbatim}
% Abir Dey's Page
	\newpage
	\section{Enumerate ABC Format and Roman Number / by: Abir Dey}
	\subsection{Introduction}
	This document demonstrates how to create enumerated lists in LaTeX using ABC format and Roman numerals, providing flexibility in list formatting.
	
	\subsection{Creating an ABC Format Enumerated List}
	To create an enumerated list in ABC format (a, b, c), use the enumerate environment with label=\alph*.:
	
	\begin{verbatim}
		\begin{enumerate}[label=\alph*.]
			\item First item
			\item Second item
			\item Third item
		\end{enumerate}
	\end{verbatim}
	
	Result:
	
	\begin{enumerate}[label=\alph*.]
		\item First item
		\item Second item
		\item Third item
	\end{enumerate}
	
	\subsection{Creating a Roman Numerals Enumerated List}
	For a list with Roman numerals (i, ii, iii), use label=\roman*.:
	
	\begin{verbatim}
		\begin{enumerate}[label=\roman*.]
			\item First item
			\item Second item
			\item Third item
		\end{enumerate}
	\end{verbatim}
	
	Result:
	
	\begin{enumerate}[label=\roman*.]
		\item First item
		\item Second item
		\item Third item
	\end{enumerate}
    \end{document}
